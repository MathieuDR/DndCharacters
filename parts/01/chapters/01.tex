% ╒══════════════════════════════════════════════════════════════════════════╕ %
% │                                CHAPTER  1                                │ %
% ╘══════════════════════════════════════════════════════════════════════════╛ %
\chapter{Sections}
\label{p:1:ch:1}

\DndDropCapLine{T}{his package is designed to aid you in} writing beautifully typeset documents for the fifth edition of the world's greatest roleplaying game. It starts by adjusting the section formatting from the defaults in \LaTeX{} to something a bit more familiar to the reader. The chapter formatting is displayed above.

\section{Section}
Sections break up chapters into large groups of associated text.

\subsection{Subsection}
Subsections further break down the information for the reader.

\subsubsection{Subsubsection}
Sub subsections are the furthest division of text that still have a block header. Below this level, headers are displayed inline.

\paragraph{Paragraph}
The paragraph format is seldom used in the core books, but is available if you prefer it to the ``normal'' style.

\subparagraph{Subparagraph}
The subparagraph format with the paragraph indent is likely going to be more familiar to the reader.

\section{Special Sections}
The module also includes functions to aid in the proper typesetting of multi-line section headers: |\DndFeatHeader| for feats, |\DndItemHeader| magic items and traps, and |\DndSpellHeader| for spells.

\label{f:TypesettingSavant}

\DndFeatHeader{Type setting savant}[Prerequisite: \LaTeX{} distribution]
You have acquired a package which aids in typesetting source material for one of your favourite games, giving you the following benefits:

\begin{itemize}
  \item You have advantage on Intelligence checks to typeset new content.
  \item When you fail an Intelligence check to typeset new content, you can ask questions online at the package's website.
  \item Really good at using \Gls{FoosQuill}
\end{itemize}

()\DndItemHeader{\Gls{FoosQuill}}{Wondrous item, rare}
This quill has 3 charges. While holding it, you can use an action to expend 1 of its charges. The quill leaps from your hand and writes a contract applicable to your situation.

The quill regains 1d3 expended charges daily at dawn.

\label{s:BeautifulTypesetting}
\DndSpellHeader%
  {Beautiful Typesetting}
  {4th-level illusion}
  {1 action}
  {5 feet}
  {S, M (ink and parchment, which the spell consumes)}
  {Until dispelled}
You are able to transform a written message of any length into a beautiful scroll. All creatures within range that can see the scroll must make a wisdom saving throw or be charmed by you until the spell ends.

While the creature is charmed by you, they cannot take their eyes off the scroll and cannot willingly move away from the scroll. Also, the targets can make a wisdom saving throw at the end of each of their turns. On a success, they are no longer charmed.

\section{Map Regions}
The map region functions |\DndArea| and |\DndSubArea| provide automatic numbering of areas.

\DndArea{Village of Hommlet}
This is the village of hommlet.

\DndSubArea{Inn of the Welcome Wench}
Inside the village is the inn of the Welcome Wench.

\DndSubArea{Blacksmith's Forge}
There's a blacksmith in town, too.

\DndArea{Foo's Castle}
This is Foo's home, a hovel of mud and sticks.

\DndSubArea{Moat}
This ditch has a board spanning it.

\DndSubArea{Entrance}
A five-foot hole reveals the dirt floor illuminated by a hole in the roof.