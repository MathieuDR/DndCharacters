% ╒══════════════════════════════════════════════════════════════════════════╕ %
% │                                CHAPTER  2                                │ %
% ╘══════════════════════════════════════════════════════════════════════════╛ %
\chapter{Text Boxes}
\label{p:1:ch:2}

The module has three environments for setting text apart so that it is drawn to the reader's attention. |DndReadAloud| is used for text that a game master would read aloud.

\begin{DndReadAloud}
  As you approach this module you get a sense that the blood and tears of many generations went into its making. A warm feeling welcomes you as you type your first words.
\end{DndReadAloud}

\section{As an Aside}
The other two environments are the |DndComment| and the |DndSidebar|. The |DndComment| is breakable and can safely be used inline in the text.

\begin{DndComment}{This Is a Comment Box!}
  A |DndComment| is a box for minimal highlighting of text. It lacks the ornamentation of |DndSidebar|, but it can handle being broken over a column.
\end{DndComment}

The |DndSidebar| is not breakable and is best used floated toward a page corner as it is below.

\begin{DndSidebar}[float=!b]{Behold the DndSidebar!}
  The |DndSidebar| is used as a sidebar. It does not break over columns and is best used with a figure environment to float it to one corner of the page where the surrounding text can then flow around it.
\end{DndSidebar}

\section{Tables}
The |DndTable| colors the even rows and is set to the width of a line by default.

\begin{DndTable}[header=Nice Table]{XX}
    Table head  & Table head \\
    Some value  & Some value \\
    Some value  & Some value \\
    Some value  & Some value
\end{DndTable}

\section{Tooltips}
Tooltips have been \tooltip****{enabled}{Only in pdf's of course!} by an extra package. You will be able to use them with the following command \begin{verbatim} \tooltip[*[*[*[*]]]]
[<link colour>]{<link text>}
[<tip box colour>]{<tip text>}
[<x-offset>,<y-offset>] \end{verbatim}

Besides that there are 5 variants. You can find them in the |packages/tooltip.sty|.